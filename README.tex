\documentclass[a4paper,10pt]{report}

\usepackage[margin=16mm]{geometry}
\usepackage{hyperref}
\usepackage{listings}
\usepackage{xcolor}

\title{Mac Hacking}
\author{Sammy Li}
\date{2017-11-12}

\hypersetup{
  colorlinks=true,
  linkcolor=blue,
  filecolor=magenta
  }
\urlstyle{same}
\lstset{
  basicstyle=\ttfamily,           % the size of the fonts that are used for the mkeyword
  numbers=left,
  tabsize=2,
  breaklines=true,
  keywordstyle=\color{magenta},          % keyword style
  commentstyle=\color{red!70!green},       % comment style
  stringstyle=\color{cyan},          % string literal style
  frame=trBL,     
  numberstyle=\footnotesize,
  morecomment=[s]{/*}{*/},
  escapeinside={(*@}{@*)},
  showspaces=false,
}
\begin{document}
\maketitle

\tableofcontents

\chapter{Foreword}
The OS for demo in this book is of macOS Sierra, Version 10.12.6.

\chapter{Awesome Softwares}
\section{Packages Manager}
\begin{itemize}
  \item \href{https://brew.sh}{Homebrew 1.3.6}
  \item \href{http://caskroom.github.io}{Homebrew-Cask 1.3.6}
\end{itemize}

\section{Homebrew}
\subsection{Sources update}
\subsubsection{Mechanism}
The sources for updating in Homebrew consists of
\begin{itemize}
  \item brew.git: basic component
  \item homebrew-core.git: core component
  \item homebrew-bottles: pre-compiled binary packages
\end{itemize}
If we want to fully update the source for updating of Homebrew, all of the three components should be replaced with sources available and faster to us.
\subsubsection{Customize sources}
The default sources for Homebrew is hosted in Github, making them not so friendly for users in some regions. Take me as an example. I'm a Chinese, and it's  usually faster for me to fetch resources from our mainland than Github. Hence, domestic sources is preferred to me. 

To customize sources for Homebrew to suit our need, we can update the remote repository ``origin'' for the 3 components to those hosted in, says Tsinghua university, as follows.
\begin{lstlisting}[language=bash]
# switch the installation directory of Homebrew
$ cd "$(brew --repo)"
# update the remote repository ``origin'' for the basic component to that of Tsinghua
$ git remote set-url origin https://mirrors.tuna.tsinghua.edu.cn/git/homebrew/brew.git

# switch the installation directory of homebrew-core
$ cd "$(brew --repo)/Library/Taps/homebrew/homebrew-core"
# update the remote repository ``origin'' for the core component to that of Tsinghua 
$ git remote set-url origin https://mirrors.tuna.tsinghua.edu.cn/git/homebrew/homebrew-core.git

# replace the domain in .bash_profile for homebrew-bottles by that of Tsinghua
$ echo 'export HOMEBREW_BOTTLE_DOMAIN=https://mirrors.tuna.tsinghua.edu.cn/homebrew-bottles' >> ~/.bash_profile

# refresh the .bash_profile to make it effective
$ source ~/.bash_profile

# don't forget to apply the update
$ brew update
\end{lstlisting}
\subsubsection{Reset to default sources}
\begin{lstlisting}[language=bash]
# switch the installation directory of Homebrew
$ cd "$(brew --repo)"
# reset the upstream repository ``origin'' for the basic component
$ git remote set-url origin https://github.com/Homebrew/brew.git

# switch the installation directory of homebrew-core
$ cd "$(brew --repo)/Library/Taps/homebrew/homebrew-core"
# reset the remote repository ``origin'' for the core component
$ git remote set-url origin https://github.com/Homebrew/homebrew-core.git

# to reset the domain for homebrew-bottle, just remove the customized one from .bash_profile

# don't forget to apply the update
$ brew update
\end{lstlisting}
\section{Homebrew-Cask}
\subsection{Customize sources}
For the similar reason to that of Homebrew, we can update the sources for updating in Homebrew-Cask to a domestic one (says, the one hosted by USTC) by commands specified as \lstlistingname~\ref{lst-homebrew-cask-src-customize}.
\begin{lstlisting}[language=bash,caption={Customize sources for Homebrew-Cask},label={lst-homebrew-cask-src-customize}]
# switch to the local repository of Homebrew-Cask
cd "$(brew --repo)"/Library/Taps/caskroom/homebrew-cask
# update its upstream
git remote set-url origin https://mirrors.ustc.edu.cn/homebrew-cask.git
\end{lstlisting}
\end{document}
